\documentclass{../tex/classe-tex3R}
\usepackage{../tex/style-tex3R}
\parametrage
\usepackage{../tex/flashcards-tex3R}
\usepackage{../tex/tnt-tex3R}

\graphicspath{{../images/}}

\tcbset{
  interieur/.style={
    top=0pt,
    right=\dimensionscellule,
    bottom=\dimensionscellule,
    left=0pt,
    before=0pt,
    before skip=0pt,
    after=0pt,
    after skip=0pt,
    left skip=0pt,
    right skip=0pt,
    colback=white,
    grow to left by=0pt,
    grow to right by=0pt,
    valign=center,
    halign=center,
    width=\dimensionsinterieur, 
    height=\dimensionsinterieur,
    arc=3mm
  },
}

\newcommand{\scaling}[1]{%
  \setlength{\dimensionscellule}{#1cm}%
  \renewcommand{\echelle}{#1}%
}%


\begin{luacode*}
  SEED = 0
  N = 4
  SCALING = 1
\end{luacode*}

\RenewEnviron{enonce}{
  \directlua{
    SEED = SEED + 1
    local enonce = '\\scaling{' .. SCALING ..'}' .. latexGrille(SEED,N,false)
    table.insert(ENONCE,enonce) 
  }
}
  
\RenewEnviron{correction}{
  \correctiontrue
  \enoncefalse
  \directlua{
    local correction = '\\scaling{' .. SCALING ..'}' .. latexGrille(SEED,N,true)
    table.insert(CORRECTION,correction) 
  }
}

\renewcommand{\difficulte}[1]{
  \directlua{
    table.insert(DIFFICULTE,"#1")
    if "#1" == "1" then
      SCALING = 1.2
      N = 4
    elseif "#1" == "2" then
      SCALING = 0.9
      N = 6
    elseif "#1" == "3" then
      SCALING = 0.75
      N = 8
    elseif "#1" == "4" then
      SCALING = 0.6
      N = 10
    else
      SCALING = 1
    end
  }
}


\begin{document}

\begin{luacode*}
  local nombreInitiation = 6
  local nombreDebutant = 9
  local nombreConfirme = 12
  local nombreExpert = 14
  local code = ''

  for _=1,nombreInitiation do
    code = code .. '\\difficulte{1}\\begin{enonce}\\end{enonce}\\begin{correction}\\end{correction}'
  end

  for _=1,nombreDebutant do
    code = code .. '\\difficulte{2}\\begin{enonce}\\end{enonce}\\begin{correction}\\end{correction}'
  end

  for _=1,nombreConfirme do
    code = code .. '\\difficulte{3}\\begin{enonce}\\end{enonce}\\begin{correction}\\end{correction}'
  end

    for _=1,nombreExpert do
    code = code .. '\\difficulte{4}\\begin{enonce}\\end{enonce}\\begin{correction}\\end{correction}'
  end

  tex.sprint(code)

\end{luacode*}



\end{document}